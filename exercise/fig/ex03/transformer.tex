\documentclass{standalone}
\usepackage{tikz}
%\usetikzlibrary{...}
\begin{document}

\begin{circuitikz}
    \coordinate (A1) at (-.3, -1,0);
    \coordinate (B1) at (4.3, -1,0);
    \coordinate (C1) at (4.3,2.7,0);
    \coordinate (D1) at (-.3,2.7,0);
                     
    \coordinate (A2) at (-.3, -1,-1);
    \coordinate (B2) at (4.3, -1,-1);
    \coordinate (C2) at (4.3,2.7,-1);
    \coordinate (D2) at (-.3,2.7,-1);
    
    \coordinate (A3) at ( .3,-.4,0);
    \coordinate (B3) at (3.7,-.4,0);
    \coordinate (C3) at (3.7,2.1,0);
    \coordinate (D3) at ( .3,2.1,0);
                     
    \coordinate (A4) at ( .3,-.4,-1);
    \coordinate (B4) at (3.7,-.4,-1);
    \coordinate (C4) at (3.7,2.1,-1);
    \coordinate (D4) at ( .3,2.1,-1);
    
  
    \coordinate (Al) at (0,-.7,-.5);
    \coordinate (Bl) at (4,-.7,-.5);
    \coordinate (Cl) at (4,2.4,-.5);
    \coordinate (Dl) at (0,2.4,-.5);
    
    % Define a formula for the coil.
        % This is what the numbers mean:
        % 0.3 ... how far the rings are apart
        % 0.4 ... how much from the side the rings are seen (try 0 and the same as the radius)
        % 1.5 ... radius of the rings
        \def\coil#1{
            {-0.75 * cos(\t * pi r)}+0.25,{0.1 * (2*#1 + \t) + 0.1*sin(\t * pi r))}
            }
    
        \def\coila#1{
            {0.75 * cos(\t * pi r)}+4.25,{0.2 * (2*#1 + \t) + 0.2*sin(\t * pi r))}
            }
        
        % Draw the part of the coil behind the rectangle
        \foreach \n in {0,1,...,6} {
            \draw[domain={0:1},smooth,variable=\t,samples=15]
                plot (\coil{\n}); 
            }
    
        \foreach \n in {0,1} {
            \draw[domain={0:1},smooth,variable=\t,samples=15]
                plot (\coila{\n}); 
            }
    
    \filldraw[fill=white,draw=black,draw opacity=0,thick,fill opacity=0.9] (A2) -- (A4) -- (D4) -- (D2) -- cycle;    
    \filldraw[fill=white,draw=black,draw opacity=0,thick,fill opacity=0.9] (B2) -- (B4) -- (C4) -- (C2) -- cycle;    
    \filldraw[fill=white,draw=black,draw opacity=0,thick,fill opacity=0.9] (A2) -- (A4) -- (B4) -- (B2) -- cycle;    
    \filldraw[fill=white,draw=black,draw opacity=0,thick,fill opacity=0.9] (D2) -- (D4) -- (C4) -- (C2) -- cycle;    
            
            
    \filldraw[fill=white,draw=black,draw opacity=1,thick,fill opacity=0] (A1) -- (A2) -- (B2) -- (B1) -- cycle;    
    \filldraw[fill=white,draw=black,draw opacity=1,thick,fill opacity=0] (A2) -- (B2) -- (C2) -- (D2) -- cycle;    
    \filldraw[fill=white,draw=black,draw opacity=1,thick,fill opacity=0] (A4) -- (B4) -- (C4) -- (D4) -- cycle;    
    \filldraw[fill=white,draw=black,draw opacity=1,thick,fill opacity=0] (A1) -- (A2) -- (D2) -- (D1) -- cycle;    
    \filldraw[fill=white,draw=black,draw opacity=1,thick,fill opacity=0] (A3) -- (A4) -- (B4) -- (B3) -- cycle;    
    \filldraw[fill=white,draw=black,draw opacity=1,thick,fill opacity=0] (B3) -- (B4) -- (C4) -- (C3) -- cycle;    
    \filldraw[fill=white,draw=black,draw opacity=1,thick,fill opacity=0] (D3) -- (D4) -- (C4) -- (C3) -- cycle;    
    \filldraw[fill=white,draw=black,draw opacity=1,thick,fill opacity=0] (A3) -- (D3) -- (D4) -- (A4) -- cycle;    
    \filldraw[fill=white,draw=black,draw opacity=1,thick,fill opacity=0] (A1) -- (B1) -- (C1) -- (D1) -- cycle;    
    
    \filldraw[fill=white,draw=black,draw opacity=1,thick,fill opacity=0.9] (B1) -- (B2) -- (C2) -- (C1) -- cycle;    
    \filldraw[fill=white,draw=black,draw opacity=1,thick,fill opacity=0.9] (D1) -- (D2) -- (C2) -- (C1) -- cycle;    
    
    \filldraw[fill=white,draw=black,draw opacity=0,thick,fill opacity=0.9] (A1) -- (A3) -- (D3) -- (D1) -- cycle;    
    \filldraw[fill=white,draw=black,draw opacity=0,thick,fill opacity=0.9] (B1) -- (B3) -- (C3) -- (C1) -- cycle;    
    \filldraw[fill=white,draw=black,draw opacity=0,thick,fill opacity=0.9] (A1) -- (A3) -- (B3) -- (B1) -- cycle;    
    \filldraw[fill=white,draw=black,draw opacity=0,thick,fill opacity=0.9] (D1) -- (D3) -- (C3) -- (C1) -- cycle;    
    
    
    \filldraw[fill=gray,draw opacity=0,fill opacity=0.8] ($(D3)+(1.2,0,0)$)--($(D4)+(1.2,0,0)$)--($(D4)+(1.2,.6,0)$)--($(D3)+(1.2,.6,0)$)--cycle;
    
    
    
        % Draw the part of the coil in front of the rectangle
        \foreach \n in {0,1,...,6} {
            \draw[domain={1:2},smooth,variable=\t,samples=15
                 ]
                plot (\coil{\n});
            }
    
        % Draw the part of the coil in front of the rectangle
        \foreach \n in {0,1} {
            \draw[domain={1:2},smooth,variable=\t,samples=15
                 ]
                plot (\coila{\n});
            }
    
    \draw[thick,dashed] (Al) --(Dl) --(Cl) --(Bl) -- cycle;
    
    \draw[>=triangle 45, <-,thick] ($(Bl)+(0,2,0)$) -- ++(1,.4,0) node[near end,anchor=west]{\ $l_\text{Fe}$};
    
    \draw[>=triangle 45, <-,thick] ($(D3)+(1.2,.5,-.7)$) -- ++(.4,1,0) node[near end,anchor=south west]{$S_{\mathrm{Fe}}$};
    
    \draw (1,0.75) node[anchor=west]{$N_1$};
    \draw (3.5,0.4) node[anchor=east]{$N_2$};
    
    \draw (-.5,1.4) to [short,i_<=$I_1$] ++(-1,0) coordinate(u1o) to [short,o-] ++(-.5,0,0) coordinate(uqo)
    (-.5,0) to [short] ++(-1,0) coordinate(u1u)  to [short,o-] ++(-.5,0,0) coordinate(uqu)
    (uqo) to [sV,v^=$U_1$] (uqu);
    
    \draw (5,0.8) to [short,-o] ++(1,0) coordinate(u2o) 
    (5,0) to [short,-o] ++(1,0) coordinate(u2u)
    (u2o) to [open,v^=$U_2$] (u2u);
    
    \end{circuitikz}





\begin{tikzpicture}
    \begin{axis}[
        /pgf/number format/.cd,
        use comma,
        1000 sep={},
        domain=0:17,
        xmin=0, xmax=17,
        ymin=0, ymax=1.7,
        samples=1000,
        axis y line=center,
         axis x line=bottom,
        xtick distance=2,
        ytick distance=0.2,
        extra y ticks=0,
        x label style={at={(axis description cs:1,0)},anchor=north west},
        y label style={at={(axis description cs:0,1)},anchor=south},
        width=0.4\textwidth,
        height=0.4\textwidth,
        xlabel={$H$ in \SI{}{\ampere \per \centi \metre}},
        ylabel={$B$ in \SI{}{\tesla}},
        grid=both,
        grid style={line width=.1pt, draw=gray!10},
        major grid style={line width=.2pt,draw=gray!50}
        ]
        % \addplot[black,domain=0:4] {x};
        \addplot[color=blue,mark=none,solid]{1.323*exp(0.003147*x)-1.323*exp(-0.4563*x)};
%          p(x) = a*exp(b*x) + c*exp(d*x)
%      Coefficients:
%        a =       1.323
%        b =    0.003147
%        c =      -1.323
%        d =     -0.4563
    \end{axis}
\end{tikzpicture}
\end{document}