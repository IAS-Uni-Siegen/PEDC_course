%%%%%%%%%%%%%%%%%%%%%%%%%%%%%%%%%%%%%%%%%%%%%%%%%%%%%%%%%%%%%
%% Begin exercise %%
%%%%%%%%%%%%%%%%%%%%%%%%%%%%%%%%%%%%%%%%%%%%%%%%%%%%%%%%%%%%%
\ex{Semiconductor properties}


%%%%%%%%%%%%%%%%%%%%%%%%%%%%%%%%%%%%%%%%%%%%%%%%%%%%%%%%%%%%%
%% Task 1 %%
%%%%%%%%%%%%%%%%%%%%%%%%%%%%%%%%%%%%%%%%%%%%%%%%%%%%%%%%%%%%%

\task{Properties of germanium as semiconductor}

The following problems explore fundamental semiconductor properties of germanium. 
Starting with atomic concentration and intrinsic resistivity, the tasks gradually introduce doping effects 
and compare the electrical behavior of intrinsic and extrinsic materials under simplified assumptions.

\begin{table}[ht]
    \centering  % Zentriert die Tabelle
    \begin{tabular}{llll}
        \toprule
         Atomic weight: &  $\SI{72.6}{\gram\per\mole}$ & Density: & $\SI{5.32}{\gram\per{\cubic\centi\meter}}$ \\ 
        \bottomrule
    \end{tabular}
    \caption{key figures of germanium.}  % Beschriftung der Tabelle
    \label{table:ex01_germanium_values}
\end{table}

\subtask{Using Avogadro’s number ($N_{Av}=\SI{6.02}{1\per\mol}$), calculate the concentration of atoms in germanium based on the numerical value given in \autoref{table:ex01_germanium_values}.}

\begin{solutionblock}
    A quantity of any substance equal to its molecular weight in grams is a mole.
    So the concentration is calculated by
    \begin{equation}
        C_{Ger} = \frac{N_{Av}}{A_\mathrm{T}} \cdot D 
        =\frac{\SI{6.02}{1\per\mol}}{\SI{72.6}{\gram\per\mole}} \cdot \SI{5.32}{\gram\per{\cubic{\centi\meter}}}
    \end{equation}
\end{solutionblock}

\subtask{Find the resistivity of intrinsic germanium at $\SI{300}{\kelvin}$ .}

\begin{solutionblock}
    Using the equation $n=p=n_i$ the conductivity of the intrinsic germanium is calculated by
    \begin{equation}
        \sigma  = n_i^2q(\mu_n+\mu_p) = \SI{2.5 \cdot 10^{13}}{atoms\per{\cubic{\centi\meter}}} \cdot \SI{1.60 \cdot 10^{-19}}{\coulomb}
        \cdot \left( \SI{3800}{\square{\centi\meter}\per{\volt\second}} + \SI{3800}{\square{\centi\meter}\per{\volt\second}} \right)
        = \SI{0.00224}{1\per{\ohm\centi\meter}}
    \end{equation}
    The resistivity results by the reciprocal value:
    \begin{equation}
        \rho = \frac{1}{\sigma} = \frac{1}{\SI{0.00224}{1\per{\ohm\centi\meter}}}= \SI{44.6}{\ohm\centi\meter}
    \end{equation}    
\end{solutionblock}


\subtask{If a donor-type impurity is added to the extent of 1 part in $10^8$ germanium atoms, find the resistivity.}

\begin{solutionblock}
    If there is 1 donor atom per $10^8$ germanium atoms, then $N_D= \SI{4.41 \cdot 10^{14}}{atoms\per{\cubic{\centi\meter}}}$.
    Using the fact $n \approx N_D$ leads to:
    \begin{equation}
        p  = \frac{n_i^2}{N_D} = \frac{\left({\SI{2.5 \cdot 10^{13}}{atoms\per{\cubic{\centi\meter}}}}\right)^2}{\SI{4.41 \cdot 10^{14}}{atoms\per{\cubic{\centi\meter}}}}
        = \SI{1.42 \cdot 10^{12}}{holes\per{\cubic{\centi\meter}}}
    \end{equation}
    Since $n \ll p$, we can neglect $p$ in calculating the conductivity. This leads to:
    \begin{equation}
        \sigma = nq\mu_n = \SI{4.41 \cdot 10^{14}}{atoms\per{\cubic{\centi\meter}}} \cdot \SI{1.60 \cdot 10^{-19}}{\coulomb}
        \cdot \SI{3800}{\square{\centi\meter}\per{\volt\second}} = \SI{0.268}{1\per{\ohm\centi\meter}}
    \end{equation}
    The resistivity results by the reciprocal value:
    \begin{equation}
        \rho = \frac{1}{\sigma} = \frac{1}{\SI{0.268}{1\per{\ohm\centi\meter}}}= \SI{3.72}{\ohm\centi\meter}
    \end{equation} 
\end{solutionblock}


\subtask{If germanium were a monovalent metal, find the ratio of its conductivity to that of the n-type semiconductor in previous subtask.}

\begin{solutionblock}
    If each atom contributed one free electron to the material the number of free electrons are $n  = 4.41 \cdot 10^{22}$.
    This leads to:
    \begin{equation}
        \sigma = nq\mu_n = \SI{4.41 \cdot 10^{22}}{atoms\per{\cubic{\centi\meter}}} \cdot \SI{1.60 \cdot 10^{-19}}{\coulomb}
        \cdot \SI{3800}{\square{\centi\meter}\per{\volt\second}} = \SI{2.58 \cdot 10^{7}}{1\per{\ohm\centi\meter}}
    \end{equation}
    The ratio yields:
    \begin{equation}
        \frac{\sigma_{metal}}{\sigma_{semi}}= \frac{ \SI{2.58 \cdot 10^{7}}{1\per{\ohm\centi\meter}}}{\SI{0.268}{1\per{\ohm\centi\meter}}}
        \approx 10^8
    \end{equation}

\end{solutionblock}




%%%%%%%%%%%%%%%%%%%%%%%%%%%%%%%%%%%%%%%%%%%%%%%%%%%%%%%%%%%%%
%% Task 2 %%
%%%%%%%%%%%%%%%%%%%%%%%%%%%%%%%%%%%%%%%%%%%%%%%%%%%%%%%%%%%%%

\task{Energy level and silicon semiconductor behaviour.} 

\subtask{Calculate the frequency of revolution for an electron in the ground state of hydrogen. For comparison, find the
frequency emitted when an electron falls from the state to the ground state.}

\subtask{Find the room-temperature resistivity of an n-type silicon doped with $10^16$ phosphorus $\si{atoms\per \cubic{\centi\meter}}$.}

\begin{solutionblock}
    At room temperature we assume that all donors are ionized. 
    This leads to:
    \begin{equation}
        n \approx N_D = \SI{10^{16}}{atoms\per{\cubic{\centi\meter}}}
    \end{equation}
    The conductivity is calculated by
    \begin{equation}
        \sigma = nq\mu_n = \SI{10^{16}}{atoms\per{\cubic{\centi\meter}}} \cdot \SI{1.60 \cdot 10^{-19}}{\coulomb}
        \cdot \SI{1300}{\square{\centi\meter}\per{\volt\second}} = \SI{20.8}{1\per{\ohm\centi\meter}}
    \end{equation}
    The resistivity results by the reciprocal value:
    \begin{equation}
        \rho = \frac{1}{\sigma} = \frac{1}{\SI{20.8}{1\per{\ohm\centi\meter}}}= \SI{0.48}{\ohm\centi\meter}
    \end{equation} 
\end{solutionblock}


\subtask{A sample of Si is doped with $10^{16}$ phosphorus $\si{atoms\per \cubic{\centi\meter}}$. 
Find the Hall voltage in a sample with $W = \SI{500}{\micro\meter}$, $A = \SI{2.5 \cdot 10^{-3}}{\square\centi\meter}$, $I = \SI{1}{\milli\ampere}$, 
and $B_z =  \SI{10^{-4}}{\weber\per\square\centi\meter}$.}

\begin{solutionblock}
    The Hall coefficient is calculated by
    \begin{equation}
        R_H = \frac{1}{qn} = \frac{1}{\SI{1.60 \cdot 10^{-19}}{\coulomb} \cdot 10^{16}} =  \SI{625}{\cubic{\centi\meter}\per\coulomb}
    \end{equation}
    The Hall voltage results from
    \begin{equation}
        V_h = R_H \frac{I}{A}B_z W = \SI{625}{\cubic{\centi\meter}\per\coulomb} \frac{\SI{1}{\milli\ampere}}{\SI{2.5 \cdot 10^{-3}}{\square\centi\meter}}
        \SI{10^{-4}}{\weber\per\square\centi\meter} \cdot \SI{500}{\micro\meter} = \SI{-1.25}{\milli\volt} 
    \end{equation}
\end{solutionblock}


\subtask{ Assume that, in an n-type semiconductor at $T = \SI{300}{\kelvin}$, the electron concentration varies linearly from 
 $\SI{10^{18}}{\cubic\centi\meter}$ to $\SI{7 \cdot 10^{17}}{\cubic\centi\meter}$ over a distance of $\SI{0.1}{\centi\meter}$.
  Calculate the diffusion current density if the electron diffusion coefficient is $D_n = \SI{1}{\square\centi\meter\per\second}$.}





\subtask{Minority carriers (holes) are injected into a homogeneous n-type semiconductor sample at one point.
 An electric field of $\SI{50}{\volt\per{\centi\meter}}$ is applied across the sample,
 and the field moves these minority carriers a distance of $\SI{1}{\centi\meter}$ in $\SI{100}{\micro\second}$.
 Find the drift velocity and the diffusivity of the minority carriers. The temperature is $\SI{300}{\kelvin}$.}





%%%%%%%%%%%%%%%%%%%%%%%%%%%%%%%%%%%%%%%%%%%%%%%%%%%%%%%%%%%%%
%% Task 3 %%
%%%%%%%%%%%%%%%%%%%%%%%%%%%%%%%%%%%%%%%%%%%%%%%%%%%%%%%%%%%%%

\task{A silicon semiconductor is to be doped for a specific application. To assess the doping quality, the resistivity of a sample is measured before and after the process.}

\subtask{Calculate the expected resistivity of silicon at $T = \SI{300}{\kelvin}$ before doping}
 
\subtask{Calculate the expected resistivity of silicon at $T = \SI{300}{\kelvin}$ after doping with arsenic impurities 
with concentration $N_d = 2\cdot \SI{10^{16}}{\cubic\centi\meter}$.}


%%%%%%%%%%%%%%%%%%%%%%%%%%%%%%%%%%%%%%%%%%%%%%%%%%%%%%%%%%%%%
%% Task 4 %%
%%%%%%%%%%%%%%%%%%%%%%%%%%%%%%%%%%%%%%%%%%%%%%%%%%%%%%%%%%%%%

\task{Hall sensor for  magnetic field measurment}

A bar of type p silicon, of thickness $d=\SI{0.5}{\milli\meter}$, with impurity concentration $N_a = \SI{10^{14}}{\cubic\centi\meter}$, is used as a Hall sensor.

\subtask{Calculate the Hall voltage for a probe current of $\SI{100}{\milli\ampere}$ and a magnetic field perpendicular to the plane of $B= \SI{0.1}{\milli\tesla}$.}

